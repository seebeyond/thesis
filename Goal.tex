\chapter{Goal}\label{chapter:goal}

In this thesis I study the problem of detecting fake reviews written by the same users under different identities and aim to use semantic similarity metrics to compute the relatedness between user reviews. This represents a new approach to opinion spam detection, meant to capture more information than a simple text similarity measure such as the cosine similarity and tap deeper into the textual particularities of a review. It is aimed at spammers which use multiple anonymous profiles to post reviews by reusing previous text they had also written, replacing the main aspect words with synonyms, while still keeping the overall sentiment the same. 

There is one obvious but important assumption - the imagination of a spammer, like any human is limited and thus will not be able to write completely different reviews about an imaginary experience every single time. Thus, it is very likely he will partially reuse some of the content and aspects between reviews. Amateur spammers will definitely reuse the same words, and here cosine and semantic similarity should more or less offer the same results. More subtle spammers will use synonyms and paraphrase and this is where the semantic similarity approach should score better. Previous work has not tackled the problem from this perspective - to the best of my knowledge no opinion spam research incorporates anything related to any semantic similarity metric or semantic proximity. 

Another shortcoming of previous research is the lack of gold-standard datasets. In order to to overcome this, researchers have employed two strategies so far - they have either used human judges to annotate fake reviews and use an agreement measure among them to decide on whether a review was fake or not, or have used a crowdsourcing tool such as the Amazon Mechanical Turk to produce known deceptive reviews. Both strategies have been found to have their weaknesses and were inaccurate or unable to generalize well on real-life data. It is fair to say that the relevant datasets are owned by large review websites, which have access to much more behavioral data, such as the user's IP, his social network data or even mouse cursor movements for that matter. This kind of information is not publicly available and thus cannot be crawled for research purposes.

This thesis proposes a complete solution to detect opinion spam of one-time reviewers using semantic similarity. It also proposes a method to detect opinion spam, using recent research models aimed at extracting product aspects from short texts, based on topic modeling and in particular on Latent Dirichlet Allocation (LDA). Another goal was to test this hypothesis on real-life reviews and make a comparison with the existing vectorial similarity models, which are based on cosine similarity. 

My hypothesis is that semantic similarity measures should outperform vector based models because they should also capture more subtle deception behavior, meaning more paraphrase intent of the spammers. Detecting fake reviews through semantic similarity methods would inherently work on users who operate in groups, know one other and paraphrase or rephrase each other's reviews inside the same group of spammers.